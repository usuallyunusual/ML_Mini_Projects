\documentclass[11pt]{article}

    \usepackage[breakable]{tcolorbox}
    \usepackage{parskip} % Stop auto-indenting (to mimic markdown behaviour)
    
    \usepackage{iftex}
    \ifPDFTeX
    	\usepackage[T1]{fontenc}
    	\usepackage{mathpazo}
    \else
    	\usepackage{fontspec}
    \fi

    % Basic figure setup, for now with no caption control since it's done
    % automatically by Pandoc (which extracts ![](path) syntax from Markdown).
    \usepackage{graphicx}
    % Maintain compatibility with old templates. Remove in nbconvert 6.0
    \let\Oldincludegraphics\includegraphics
    % Ensure that by default, figures have no caption (until we provide a
    % proper Figure object with a Caption API and a way to capture that
    % in the conversion process - todo).
    \usepackage{caption}
    \DeclareCaptionFormat{nocaption}{}
    \captionsetup{format=nocaption,aboveskip=0pt,belowskip=0pt}

    \usepackage[Export]{adjustbox} % Used to constrain images to a maximum size
    \adjustboxset{max size={0.9\linewidth}{0.9\paperheight}}
    \usepackage{float}
    \floatplacement{figure}{H} % forces figures to be placed at the correct location
    \usepackage{xcolor} % Allow colors to be defined
    \usepackage{enumerate} % Needed for markdown enumerations to work
    \usepackage{geometry} % Used to adjust the document margins
    \usepackage{amsmath} % Equations
    \usepackage{amssymb} % Equations
    \usepackage{textcomp} % defines textquotesingle
    % Hack from http://tex.stackexchange.com/a/47451/13684:
    \AtBeginDocument{%
        \def\PYZsq{\textquotesingle}% Upright quotes in Pygmentized code
    }
    \usepackage{upquote} % Upright quotes for verbatim code
    \usepackage{eurosym} % defines \euro
    \usepackage[mathletters]{ucs} % Extended unicode (utf-8) support
    \usepackage{fancyvrb} % verbatim replacement that allows latex
    \usepackage{grffile} % extends the file name processing of package graphics 
                         % to support a larger range
    \makeatletter % fix for grffile with XeLaTeX
    \def\Gread@@xetex#1{%
      \IfFileExists{"\Gin@base".bb}%
      {\Gread@eps{\Gin@base.bb}}%
      {\Gread@@xetex@aux#1}%
    }
    \makeatother

    % The hyperref package gives us a pdf with properly built
    % internal navigation ('pdf bookmarks' for the table of contents,
    % internal cross-reference links, web links for URLs, etc.)
    \usepackage{hyperref}
    % The default LaTeX title has an obnoxious amount of whitespace. By default,
    % titling removes some of it. It also provides customization options.
    \usepackage{titling}
    \usepackage{longtable} % longtable support required by pandoc >1.10
    \usepackage{booktabs}  % table support for pandoc > 1.12.2
    \usepackage[inline]{enumitem} % IRkernel/repr support (it uses the enumerate* environment)
    \usepackage[normalem]{ulem} % ulem is needed to support strikethroughs (\sout)
                                % normalem makes italics be italics, not underlines
    \usepackage{mathrsfs}
    

    
    % Colors for the hyperref package
    \definecolor{urlcolor}{rgb}{0,.145,.698}
    \definecolor{linkcolor}{rgb}{.71,0.21,0.01}
    \definecolor{citecolor}{rgb}{.12,.54,.11}

    % ANSI colors
    \definecolor{ansi-black}{HTML}{3E424D}
    \definecolor{ansi-black-intense}{HTML}{282C36}
    \definecolor{ansi-red}{HTML}{E75C58}
    \definecolor{ansi-red-intense}{HTML}{B22B31}
    \definecolor{ansi-green}{HTML}{00A250}
    \definecolor{ansi-green-intense}{HTML}{007427}
    \definecolor{ansi-yellow}{HTML}{DDB62B}
    \definecolor{ansi-yellow-intense}{HTML}{B27D12}
    \definecolor{ansi-blue}{HTML}{208FFB}
    \definecolor{ansi-blue-intense}{HTML}{0065CA}
    \definecolor{ansi-magenta}{HTML}{D160C4}
    \definecolor{ansi-magenta-intense}{HTML}{A03196}
    \definecolor{ansi-cyan}{HTML}{60C6C8}
    \definecolor{ansi-cyan-intense}{HTML}{258F8F}
    \definecolor{ansi-white}{HTML}{C5C1B4}
    \definecolor{ansi-white-intense}{HTML}{A1A6B2}
    \definecolor{ansi-default-inverse-fg}{HTML}{FFFFFF}
    \definecolor{ansi-default-inverse-bg}{HTML}{000000}

    % commands and environments needed by pandoc snippets
    % extracted from the output of `pandoc -s`
    \providecommand{\tightlist}{%
      \setlength{\itemsep}{0pt}\setlength{\parskip}{0pt}}
    \DefineVerbatimEnvironment{Highlighting}{Verbatim}{commandchars=\\\{\}}
    % Add ',fontsize=\small' for more characters per line
    \newenvironment{Shaded}{}{}
    \newcommand{\KeywordTok}[1]{\textcolor[rgb]{0.00,0.44,0.13}{\textbf{{#1}}}}
    \newcommand{\DataTypeTok}[1]{\textcolor[rgb]{0.56,0.13,0.00}{{#1}}}
    \newcommand{\DecValTok}[1]{\textcolor[rgb]{0.25,0.63,0.44}{{#1}}}
    \newcommand{\BaseNTok}[1]{\textcolor[rgb]{0.25,0.63,0.44}{{#1}}}
    \newcommand{\FloatTok}[1]{\textcolor[rgb]{0.25,0.63,0.44}{{#1}}}
    \newcommand{\CharTok}[1]{\textcolor[rgb]{0.25,0.44,0.63}{{#1}}}
    \newcommand{\StringTok}[1]{\textcolor[rgb]{0.25,0.44,0.63}{{#1}}}
    \newcommand{\CommentTok}[1]{\textcolor[rgb]{0.38,0.63,0.69}{\textit{{#1}}}}
    \newcommand{\OtherTok}[1]{\textcolor[rgb]{0.00,0.44,0.13}{{#1}}}
    \newcommand{\AlertTok}[1]{\textcolor[rgb]{1.00,0.00,0.00}{\textbf{{#1}}}}
    \newcommand{\FunctionTok}[1]{\textcolor[rgb]{0.02,0.16,0.49}{{#1}}}
    \newcommand{\RegionMarkerTok}[1]{{#1}}
    \newcommand{\ErrorTok}[1]{\textcolor[rgb]{1.00,0.00,0.00}{\textbf{{#1}}}}
    \newcommand{\NormalTok}[1]{{#1}}
    
    % Additional commands for more recent versions of Pandoc
    \newcommand{\ConstantTok}[1]{\textcolor[rgb]{0.53,0.00,0.00}{{#1}}}
    \newcommand{\SpecialCharTok}[1]{\textcolor[rgb]{0.25,0.44,0.63}{{#1}}}
    \newcommand{\VerbatimStringTok}[1]{\textcolor[rgb]{0.25,0.44,0.63}{{#1}}}
    \newcommand{\SpecialStringTok}[1]{\textcolor[rgb]{0.73,0.40,0.53}{{#1}}}
    \newcommand{\ImportTok}[1]{{#1}}
    \newcommand{\DocumentationTok}[1]{\textcolor[rgb]{0.73,0.13,0.13}{\textit{{#1}}}}
    \newcommand{\AnnotationTok}[1]{\textcolor[rgb]{0.38,0.63,0.69}{\textbf{\textit{{#1}}}}}
    \newcommand{\CommentVarTok}[1]{\textcolor[rgb]{0.38,0.63,0.69}{\textbf{\textit{{#1}}}}}
    \newcommand{\VariableTok}[1]{\textcolor[rgb]{0.10,0.09,0.49}{{#1}}}
    \newcommand{\ControlFlowTok}[1]{\textcolor[rgb]{0.00,0.44,0.13}{\textbf{{#1}}}}
    \newcommand{\OperatorTok}[1]{\textcolor[rgb]{0.40,0.40,0.40}{{#1}}}
    \newcommand{\BuiltInTok}[1]{{#1}}
    \newcommand{\ExtensionTok}[1]{{#1}}
    \newcommand{\PreprocessorTok}[1]{\textcolor[rgb]{0.74,0.48,0.00}{{#1}}}
    \newcommand{\AttributeTok}[1]{\textcolor[rgb]{0.49,0.56,0.16}{{#1}}}
    \newcommand{\InformationTok}[1]{\textcolor[rgb]{0.38,0.63,0.69}{\textbf{\textit{{#1}}}}}
    \newcommand{\WarningTok}[1]{\textcolor[rgb]{0.38,0.63,0.69}{\textbf{\textit{{#1}}}}}
    
    
    % Define a nice break command that doesn't care if a line doesn't already
    % exist.
    \def\br{\hspace*{\fill} \\* }
    % Math Jax compatibility definitions
    \def\gt{>}
    \def\lt{<}
    \let\Oldtex\TeX
    \let\Oldlatex\LaTeX
    \renewcommand{\TeX}{\textrm{\Oldtex}}
    \renewcommand{\LaTeX}{\textrm{\Oldlatex}}
    % Document parameters
    % Document title
    \title{ML\_prog2}
    
    
    
    
    
% Pygments definitions
\makeatletter
\def\PY@reset{\let\PY@it=\relax \let\PY@bf=\relax%
    \let\PY@ul=\relax \let\PY@tc=\relax%
    \let\PY@bc=\relax \let\PY@ff=\relax}
\def\PY@tok#1{\csname PY@tok@#1\endcsname}
\def\PY@toks#1+{\ifx\relax#1\empty\else%
    \PY@tok{#1}\expandafter\PY@toks\fi}
\def\PY@do#1{\PY@bc{\PY@tc{\PY@ul{%
    \PY@it{\PY@bf{\PY@ff{#1}}}}}}}
\def\PY#1#2{\PY@reset\PY@toks#1+\relax+\PY@do{#2}}

\expandafter\def\csname PY@tok@w\endcsname{\def\PY@tc##1{\textcolor[rgb]{0.73,0.73,0.73}{##1}}}
\expandafter\def\csname PY@tok@c\endcsname{\let\PY@it=\textit\def\PY@tc##1{\textcolor[rgb]{0.25,0.50,0.50}{##1}}}
\expandafter\def\csname PY@tok@cp\endcsname{\def\PY@tc##1{\textcolor[rgb]{0.74,0.48,0.00}{##1}}}
\expandafter\def\csname PY@tok@k\endcsname{\let\PY@bf=\textbf\def\PY@tc##1{\textcolor[rgb]{0.00,0.50,0.00}{##1}}}
\expandafter\def\csname PY@tok@kp\endcsname{\def\PY@tc##1{\textcolor[rgb]{0.00,0.50,0.00}{##1}}}
\expandafter\def\csname PY@tok@kt\endcsname{\def\PY@tc##1{\textcolor[rgb]{0.69,0.00,0.25}{##1}}}
\expandafter\def\csname PY@tok@o\endcsname{\def\PY@tc##1{\textcolor[rgb]{0.40,0.40,0.40}{##1}}}
\expandafter\def\csname PY@tok@ow\endcsname{\let\PY@bf=\textbf\def\PY@tc##1{\textcolor[rgb]{0.67,0.13,1.00}{##1}}}
\expandafter\def\csname PY@tok@nb\endcsname{\def\PY@tc##1{\textcolor[rgb]{0.00,0.50,0.00}{##1}}}
\expandafter\def\csname PY@tok@nf\endcsname{\def\PY@tc##1{\textcolor[rgb]{0.00,0.00,1.00}{##1}}}
\expandafter\def\csname PY@tok@nc\endcsname{\let\PY@bf=\textbf\def\PY@tc##1{\textcolor[rgb]{0.00,0.00,1.00}{##1}}}
\expandafter\def\csname PY@tok@nn\endcsname{\let\PY@bf=\textbf\def\PY@tc##1{\textcolor[rgb]{0.00,0.00,1.00}{##1}}}
\expandafter\def\csname PY@tok@ne\endcsname{\let\PY@bf=\textbf\def\PY@tc##1{\textcolor[rgb]{0.82,0.25,0.23}{##1}}}
\expandafter\def\csname PY@tok@nv\endcsname{\def\PY@tc##1{\textcolor[rgb]{0.10,0.09,0.49}{##1}}}
\expandafter\def\csname PY@tok@no\endcsname{\def\PY@tc##1{\textcolor[rgb]{0.53,0.00,0.00}{##1}}}
\expandafter\def\csname PY@tok@nl\endcsname{\def\PY@tc##1{\textcolor[rgb]{0.63,0.63,0.00}{##1}}}
\expandafter\def\csname PY@tok@ni\endcsname{\let\PY@bf=\textbf\def\PY@tc##1{\textcolor[rgb]{0.60,0.60,0.60}{##1}}}
\expandafter\def\csname PY@tok@na\endcsname{\def\PY@tc##1{\textcolor[rgb]{0.49,0.56,0.16}{##1}}}
\expandafter\def\csname PY@tok@nt\endcsname{\let\PY@bf=\textbf\def\PY@tc##1{\textcolor[rgb]{0.00,0.50,0.00}{##1}}}
\expandafter\def\csname PY@tok@nd\endcsname{\def\PY@tc##1{\textcolor[rgb]{0.67,0.13,1.00}{##1}}}
\expandafter\def\csname PY@tok@s\endcsname{\def\PY@tc##1{\textcolor[rgb]{0.73,0.13,0.13}{##1}}}
\expandafter\def\csname PY@tok@sd\endcsname{\let\PY@it=\textit\def\PY@tc##1{\textcolor[rgb]{0.73,0.13,0.13}{##1}}}
\expandafter\def\csname PY@tok@si\endcsname{\let\PY@bf=\textbf\def\PY@tc##1{\textcolor[rgb]{0.73,0.40,0.53}{##1}}}
\expandafter\def\csname PY@tok@se\endcsname{\let\PY@bf=\textbf\def\PY@tc##1{\textcolor[rgb]{0.73,0.40,0.13}{##1}}}
\expandafter\def\csname PY@tok@sr\endcsname{\def\PY@tc##1{\textcolor[rgb]{0.73,0.40,0.53}{##1}}}
\expandafter\def\csname PY@tok@ss\endcsname{\def\PY@tc##1{\textcolor[rgb]{0.10,0.09,0.49}{##1}}}
\expandafter\def\csname PY@tok@sx\endcsname{\def\PY@tc##1{\textcolor[rgb]{0.00,0.50,0.00}{##1}}}
\expandafter\def\csname PY@tok@m\endcsname{\def\PY@tc##1{\textcolor[rgb]{0.40,0.40,0.40}{##1}}}
\expandafter\def\csname PY@tok@gh\endcsname{\let\PY@bf=\textbf\def\PY@tc##1{\textcolor[rgb]{0.00,0.00,0.50}{##1}}}
\expandafter\def\csname PY@tok@gu\endcsname{\let\PY@bf=\textbf\def\PY@tc##1{\textcolor[rgb]{0.50,0.00,0.50}{##1}}}
\expandafter\def\csname PY@tok@gd\endcsname{\def\PY@tc##1{\textcolor[rgb]{0.63,0.00,0.00}{##1}}}
\expandafter\def\csname PY@tok@gi\endcsname{\def\PY@tc##1{\textcolor[rgb]{0.00,0.63,0.00}{##1}}}
\expandafter\def\csname PY@tok@gr\endcsname{\def\PY@tc##1{\textcolor[rgb]{1.00,0.00,0.00}{##1}}}
\expandafter\def\csname PY@tok@ge\endcsname{\let\PY@it=\textit}
\expandafter\def\csname PY@tok@gs\endcsname{\let\PY@bf=\textbf}
\expandafter\def\csname PY@tok@gp\endcsname{\let\PY@bf=\textbf\def\PY@tc##1{\textcolor[rgb]{0.00,0.00,0.50}{##1}}}
\expandafter\def\csname PY@tok@go\endcsname{\def\PY@tc##1{\textcolor[rgb]{0.53,0.53,0.53}{##1}}}
\expandafter\def\csname PY@tok@gt\endcsname{\def\PY@tc##1{\textcolor[rgb]{0.00,0.27,0.87}{##1}}}
\expandafter\def\csname PY@tok@err\endcsname{\def\PY@bc##1{\setlength{\fboxsep}{0pt}\fcolorbox[rgb]{1.00,0.00,0.00}{1,1,1}{\strut ##1}}}
\expandafter\def\csname PY@tok@kc\endcsname{\let\PY@bf=\textbf\def\PY@tc##1{\textcolor[rgb]{0.00,0.50,0.00}{##1}}}
\expandafter\def\csname PY@tok@kd\endcsname{\let\PY@bf=\textbf\def\PY@tc##1{\textcolor[rgb]{0.00,0.50,0.00}{##1}}}
\expandafter\def\csname PY@tok@kn\endcsname{\let\PY@bf=\textbf\def\PY@tc##1{\textcolor[rgb]{0.00,0.50,0.00}{##1}}}
\expandafter\def\csname PY@tok@kr\endcsname{\let\PY@bf=\textbf\def\PY@tc##1{\textcolor[rgb]{0.00,0.50,0.00}{##1}}}
\expandafter\def\csname PY@tok@bp\endcsname{\def\PY@tc##1{\textcolor[rgb]{0.00,0.50,0.00}{##1}}}
\expandafter\def\csname PY@tok@fm\endcsname{\def\PY@tc##1{\textcolor[rgb]{0.00,0.00,1.00}{##1}}}
\expandafter\def\csname PY@tok@vc\endcsname{\def\PY@tc##1{\textcolor[rgb]{0.10,0.09,0.49}{##1}}}
\expandafter\def\csname PY@tok@vg\endcsname{\def\PY@tc##1{\textcolor[rgb]{0.10,0.09,0.49}{##1}}}
\expandafter\def\csname PY@tok@vi\endcsname{\def\PY@tc##1{\textcolor[rgb]{0.10,0.09,0.49}{##1}}}
\expandafter\def\csname PY@tok@vm\endcsname{\def\PY@tc##1{\textcolor[rgb]{0.10,0.09,0.49}{##1}}}
\expandafter\def\csname PY@tok@sa\endcsname{\def\PY@tc##1{\textcolor[rgb]{0.73,0.13,0.13}{##1}}}
\expandafter\def\csname PY@tok@sb\endcsname{\def\PY@tc##1{\textcolor[rgb]{0.73,0.13,0.13}{##1}}}
\expandafter\def\csname PY@tok@sc\endcsname{\def\PY@tc##1{\textcolor[rgb]{0.73,0.13,0.13}{##1}}}
\expandafter\def\csname PY@tok@dl\endcsname{\def\PY@tc##1{\textcolor[rgb]{0.73,0.13,0.13}{##1}}}
\expandafter\def\csname PY@tok@s2\endcsname{\def\PY@tc##1{\textcolor[rgb]{0.73,0.13,0.13}{##1}}}
\expandafter\def\csname PY@tok@sh\endcsname{\def\PY@tc##1{\textcolor[rgb]{0.73,0.13,0.13}{##1}}}
\expandafter\def\csname PY@tok@s1\endcsname{\def\PY@tc##1{\textcolor[rgb]{0.73,0.13,0.13}{##1}}}
\expandafter\def\csname PY@tok@mb\endcsname{\def\PY@tc##1{\textcolor[rgb]{0.40,0.40,0.40}{##1}}}
\expandafter\def\csname PY@tok@mf\endcsname{\def\PY@tc##1{\textcolor[rgb]{0.40,0.40,0.40}{##1}}}
\expandafter\def\csname PY@tok@mh\endcsname{\def\PY@tc##1{\textcolor[rgb]{0.40,0.40,0.40}{##1}}}
\expandafter\def\csname PY@tok@mi\endcsname{\def\PY@tc##1{\textcolor[rgb]{0.40,0.40,0.40}{##1}}}
\expandafter\def\csname PY@tok@il\endcsname{\def\PY@tc##1{\textcolor[rgb]{0.40,0.40,0.40}{##1}}}
\expandafter\def\csname PY@tok@mo\endcsname{\def\PY@tc##1{\textcolor[rgb]{0.40,0.40,0.40}{##1}}}
\expandafter\def\csname PY@tok@ch\endcsname{\let\PY@it=\textit\def\PY@tc##1{\textcolor[rgb]{0.25,0.50,0.50}{##1}}}
\expandafter\def\csname PY@tok@cm\endcsname{\let\PY@it=\textit\def\PY@tc##1{\textcolor[rgb]{0.25,0.50,0.50}{##1}}}
\expandafter\def\csname PY@tok@cpf\endcsname{\let\PY@it=\textit\def\PY@tc##1{\textcolor[rgb]{0.25,0.50,0.50}{##1}}}
\expandafter\def\csname PY@tok@c1\endcsname{\let\PY@it=\textit\def\PY@tc##1{\textcolor[rgb]{0.25,0.50,0.50}{##1}}}
\expandafter\def\csname PY@tok@cs\endcsname{\let\PY@it=\textit\def\PY@tc##1{\textcolor[rgb]{0.25,0.50,0.50}{##1}}}

\def\PYZbs{\char`\\}
\def\PYZus{\char`\_}
\def\PYZob{\char`\{}
\def\PYZcb{\char`\}}
\def\PYZca{\char`\^}
\def\PYZam{\char`\&}
\def\PYZlt{\char`\<}
\def\PYZgt{\char`\>}
\def\PYZsh{\char`\#}
\def\PYZpc{\char`\%}
\def\PYZdl{\char`\$}
\def\PYZhy{\char`\-}
\def\PYZsq{\char`\'}
\def\PYZdq{\char`\"}
\def\PYZti{\char`\~}
% for compatibility with earlier versions
\def\PYZat{@}
\def\PYZlb{[}
\def\PYZrb{]}
\makeatother


    % For linebreaks inside Verbatim environment from package fancyvrb. 
    \makeatletter
        \newbox\Wrappedcontinuationbox 
        \newbox\Wrappedvisiblespacebox 
        \newcommand*\Wrappedvisiblespace {\textcolor{red}{\textvisiblespace}} 
        \newcommand*\Wrappedcontinuationsymbol {\textcolor{red}{\llap{\tiny$\m@th\hookrightarrow$}}} 
        \newcommand*\Wrappedcontinuationindent {3ex } 
        \newcommand*\Wrappedafterbreak {\kern\Wrappedcontinuationindent\copy\Wrappedcontinuationbox} 
        % Take advantage of the already applied Pygments mark-up to insert 
        % potential linebreaks for TeX processing. 
        %        {, <, #, %, $, ' and ": go to next line. 
        %        _, }, ^, &, >, - and ~: stay at end of broken line. 
        % Use of \textquotesingle for straight quote. 
        \newcommand*\Wrappedbreaksatspecials {% 
            \def\PYGZus{\discretionary{\char`\_}{\Wrappedafterbreak}{\char`\_}}% 
            \def\PYGZob{\discretionary{}{\Wrappedafterbreak\char`\{}{\char`\{}}% 
            \def\PYGZcb{\discretionary{\char`\}}{\Wrappedafterbreak}{\char`\}}}% 
            \def\PYGZca{\discretionary{\char`\^}{\Wrappedafterbreak}{\char`\^}}% 
            \def\PYGZam{\discretionary{\char`\&}{\Wrappedafterbreak}{\char`\&}}% 
            \def\PYGZlt{\discretionary{}{\Wrappedafterbreak\char`\<}{\char`\<}}% 
            \def\PYGZgt{\discretionary{\char`\>}{\Wrappedafterbreak}{\char`\>}}% 
            \def\PYGZsh{\discretionary{}{\Wrappedafterbreak\char`\#}{\char`\#}}% 
            \def\PYGZpc{\discretionary{}{\Wrappedafterbreak\char`\%}{\char`\%}}% 
            \def\PYGZdl{\discretionary{}{\Wrappedafterbreak\char`\$}{\char`\$}}% 
            \def\PYGZhy{\discretionary{\char`\-}{\Wrappedafterbreak}{\char`\-}}% 
            \def\PYGZsq{\discretionary{}{\Wrappedafterbreak\textquotesingle}{\textquotesingle}}% 
            \def\PYGZdq{\discretionary{}{\Wrappedafterbreak\char`\"}{\char`\"}}% 
            \def\PYGZti{\discretionary{\char`\~}{\Wrappedafterbreak}{\char`\~}}% 
        } 
        % Some characters . , ; ? ! / are not pygmentized. 
        % This macro makes them "active" and they will insert potential linebreaks 
        \newcommand*\Wrappedbreaksatpunct {% 
            \lccode`\~`\.\lowercase{\def~}{\discretionary{\hbox{\char`\.}}{\Wrappedafterbreak}{\hbox{\char`\.}}}% 
            \lccode`\~`\,\lowercase{\def~}{\discretionary{\hbox{\char`\,}}{\Wrappedafterbreak}{\hbox{\char`\,}}}% 
            \lccode`\~`\;\lowercase{\def~}{\discretionary{\hbox{\char`\;}}{\Wrappedafterbreak}{\hbox{\char`\;}}}% 
            \lccode`\~`\:\lowercase{\def~}{\discretionary{\hbox{\char`\:}}{\Wrappedafterbreak}{\hbox{\char`\:}}}% 
            \lccode`\~`\?\lowercase{\def~}{\discretionary{\hbox{\char`\?}}{\Wrappedafterbreak}{\hbox{\char`\?}}}% 
            \lccode`\~`\!\lowercase{\def~}{\discretionary{\hbox{\char`\!}}{\Wrappedafterbreak}{\hbox{\char`\!}}}% 
            \lccode`\~`\/\lowercase{\def~}{\discretionary{\hbox{\char`\/}}{\Wrappedafterbreak}{\hbox{\char`\/}}}% 
            \catcode`\.\active
            \catcode`\,\active 
            \catcode`\;\active
            \catcode`\:\active
            \catcode`\?\active
            \catcode`\!\active
            \catcode`\/\active 
            \lccode`\~`\~ 	
        }
    \makeatother

    \let\OriginalVerbatim=\Verbatim
    \makeatletter
    \renewcommand{\Verbatim}[1][1]{%
        %\parskip\z@skip
        \sbox\Wrappedcontinuationbox {\Wrappedcontinuationsymbol}%
        \sbox\Wrappedvisiblespacebox {\FV@SetupFont\Wrappedvisiblespace}%
        \def\FancyVerbFormatLine ##1{\hsize\linewidth
            \vtop{\raggedright\hyphenpenalty\z@\exhyphenpenalty\z@
                \doublehyphendemerits\z@\finalhyphendemerits\z@
                \strut ##1\strut}%
        }%
        % If the linebreak is at a space, the latter will be displayed as visible
        % space at end of first line, and a continuation symbol starts next line.
        % Stretch/shrink are however usually zero for typewriter font.
        \def\FV@Space {%
            \nobreak\hskip\z@ plus\fontdimen3\font minus\fontdimen4\font
            \discretionary{\copy\Wrappedvisiblespacebox}{\Wrappedafterbreak}
            {\kern\fontdimen2\font}%
        }%
        
        % Allow breaks at special characters using \PYG... macros.
        \Wrappedbreaksatspecials
        % Breaks at punctuation characters . , ; ? ! and / need catcode=\active 	
        \OriginalVerbatim[#1,codes*=\Wrappedbreaksatpunct]%
    }
    \makeatother

    % Exact colors from NB
    \definecolor{incolor}{HTML}{303F9F}
    \definecolor{outcolor}{HTML}{D84315}
    \definecolor{cellborder}{HTML}{CFCFCF}
    \definecolor{cellbackground}{HTML}{F7F7F7}
    
    % prompt
    \makeatletter
    \newcommand{\boxspacing}{\kern\kvtcb@left@rule\kern\kvtcb@boxsep}
    \makeatother
    \newcommand{\prompt}[4]{
        \ttfamily\llap{{\color{#2}[#3]:\hspace{3pt}#4}}\vspace{-\baselineskip}
    }
    

    
    % Prevent overflowing lines due to hard-to-break entities
    \sloppy 
    % Setup hyperref package
    \hypersetup{
      breaklinks=true,  % so long urls are correctly broken across lines
      colorlinks=true,
      urlcolor=urlcolor,
      linkcolor=linkcolor,
      citecolor=citecolor,
      }
    % Slightly bigger margins than the latex defaults
    
    \geometry{verbose,tmargin=1in,bmargin=1in,lmargin=1in,rmargin=1in}
    
    

\begin{document}
    
    \maketitle
    
    

    
    \hypertarget{machine-learning-lab-2}{%
\section{Machine Learning Lab 2}\label{machine-learning-lab-2}}

\hypertarget{importing-libraries-for-entire-exercise}{%
\subsection{Importing libraries for entire
exercise}\label{importing-libraries-for-entire-exercise}}

    \begin{tcolorbox}[breakable, size=fbox, boxrule=1pt, pad at break*=1mm,colback=cellbackground, colframe=cellborder]
\prompt{In}{incolor}{24}{\boxspacing}
\begin{Verbatim}[commandchars=\\\{\}]
\PY{k+kn}{import} \PY{n+nn}{pandas} \PY{k}{as} \PY{n+nn}{pd}
\PY{k+kn}{import} \PY{n+nn}{numpy} \PY{k}{as} \PY{n+nn}{np}
\PY{k+kn}{from} \PY{n+nn}{sklearn}\PY{n+nn}{.}\PY{n+nn}{model\PYZus{}selection} \PY{k+kn}{import} \PY{n}{train\PYZus{}test\PYZus{}split}
\PY{k+kn}{from} \PY{n+nn}{sklearn}\PY{n+nn}{.}\PY{n+nn}{naive\PYZus{}bayes} \PY{k+kn}{import} \PY{n}{GaussianNB}
\PY{k+kn}{from} \PY{n+nn}{sklearn} \PY{k+kn}{import} \PY{n}{metrics}
\PY{k+kn}{import} \PY{n+nn}{matplotlib}\PY{n+nn}{.}\PY{n+nn}{pyplot} \PY{k}{as} \PY{n+nn}{plt}
\PY{k+kn}{from} \PY{n+nn}{sklearn}\PY{n+nn}{.}\PY{n+nn}{metrics} \PY{k+kn}{import} \PY{n}{plot\PYZus{}precision\PYZus{}recall\PYZus{}curve}
\PY{k+kn}{from} \PY{n+nn}{sklearn}\PY{n+nn}{.}\PY{n+nn}{metrics} \PY{k+kn}{import} \PY{n}{plot\PYZus{}confusion\PYZus{}matrix}
\PY{k+kn}{from} \PY{n+nn}{sklearn}\PY{n+nn}{.}\PY{n+nn}{feature\PYZus{}extraction}\PY{n+nn}{.}\PY{n+nn}{text} \PY{k+kn}{import} \PY{n}{TfidfVectorizer}
\PY{k+kn}{from} \PY{n+nn}{sklearn}\PY{n+nn}{.}\PY{n+nn}{naive\PYZus{}bayes} \PY{k+kn}{import} \PY{n}{MultinomialNB}
\PY{k+kn}{from} \PY{n+nn}{sklearn}\PY{n+nn}{.}\PY{n+nn}{pipeline} \PY{k+kn}{import} \PY{n}{make\PYZus{}pipeline}
\PY{k+kn}{from} \PY{n+nn}{sklearn}\PY{n+nn}{.}\PY{n+nn}{metrics} \PY{k+kn}{import} \PY{n}{confusion\PYZus{}matrix}
\PY{k+kn}{import} \PY{n+nn}{seaborn} \PY{k}{as} \PY{n+nn}{sns}\PY{p}{;} \PY{n}{sns}\PY{o}{.}\PY{n}{set}\PY{p}{(}\PY{p}{)}

\PY{k+kn}{from} \PY{n+nn}{sklearn}\PY{n+nn}{.}\PY{n+nn}{datasets} \PY{k+kn}{import} \PY{n}{fetch\PYZus{}20newsgroups}
\PY{o}{\PYZpc{}}\PY{k}{matplotlib} inline
\end{Verbatim}
\end{tcolorbox}

    \hypertarget{naive-bayes-classification-on-numeric-data}{%
\section{Naive Bayes Classification on numeric
data}\label{naive-bayes-classification-on-numeric-data}}

\hypertarget{loading-data}{%
\subsection{Loading data}\label{loading-data}}

    \begin{tcolorbox}[breakable, size=fbox, boxrule=1pt, pad at break*=1mm,colback=cellbackground, colframe=cellborder]
\prompt{In}{incolor}{25}{\boxspacing}
\begin{Verbatim}[commandchars=\\\{\}]
\PY{n}{data} \PY{o}{=} \PY{n}{pd}\PY{o}{.}\PY{n}{read\PYZus{}csv}\PY{p}{(}\PY{l+s+s1}{\PYZsq{}}\PY{l+s+s1}{data.csv}\PY{l+s+s1}{\PYZsq{}}\PY{p}{)}
\PY{n}{data\PYZus{}sec} \PY{o}{=} \PY{n}{pd}\PY{o}{.}\PY{n}{read\PYZus{}csv}\PY{p}{(}\PY{l+s+s1}{\PYZsq{}}\PY{l+s+s1}{data.csv}\PY{l+s+s1}{\PYZsq{}}\PY{p}{)}
\end{Verbatim}
\end{tcolorbox}

    \hypertarget{defining-function-to-convert-classes-to-numeric}{%
\subsection{Defining function to convert classes to
numeric}\label{defining-function-to-convert-classes-to-numeric}}

    \begin{tcolorbox}[breakable, size=fbox, boxrule=1pt, pad at break*=1mm,colback=cellbackground, colframe=cellborder]
\prompt{In}{incolor}{26}{\boxspacing}
\begin{Verbatim}[commandchars=\\\{\}]
\PY{k}{def} \PY{n+nf}{diagnum}\PY{p}{(}\PY{n}{x}\PY{p}{)}\PY{p}{:}
    \PY{k}{if} \PY{n}{x} \PY{o}{==} \PY{l+s+s1}{\PYZsq{}}\PY{l+s+s1}{M}\PY{l+s+s1}{\PYZsq{}} \PY{o+ow}{or} \PY{n}{x} \PY{o}{==} \PY{l+m+mi}{1}\PY{p}{:}
        \PY{k}{return} \PY{l+m+mi}{1}
    \PY{k}{else}\PY{p}{:}
        \PY{k}{return} \PY{l+m+mi}{0}
\end{Verbatim}
\end{tcolorbox}

    \hypertarget{applying-conversion-to-entire-dataframe-and-also-preparing-output-vector}{%
\subsection{Applying conversion to entire dataframe and also preparing
output
vector}\label{applying-conversion-to-entire-dataframe-and-also-preparing-output-vector}}

    \begin{tcolorbox}[breakable, size=fbox, boxrule=1pt, pad at break*=1mm,colback=cellbackground, colframe=cellborder]
\prompt{In}{incolor}{27}{\boxspacing}
\begin{Verbatim}[commandchars=\\\{\}]
\PY{n}{data\PYZus{}sec\PYZus{}y} \PY{o}{=} \PY{n}{data\PYZus{}sec}\PY{p}{[}\PY{l+s+s1}{\PYZsq{}}\PY{l+s+s1}{diagnosis}\PY{l+s+s1}{\PYZsq{}}\PY{p}{]}\PY{o}{.}\PY{n}{apply}\PY{p}{(}\PY{n}{diagnum}\PY{p}{)}
\PY{n}{data\PYZus{}sec} \PY{o}{=} \PY{n}{data\PYZus{}sec}\PY{o}{.}\PY{n}{drop}\PY{p}{(}\PY{p}{[}\PY{l+s+s1}{\PYZsq{}}\PY{l+s+s1}{diagnosis}\PY{l+s+s1}{\PYZsq{}}\PY{p}{]}\PY{p}{,}\PY{n}{axis} \PY{o}{=} \PY{l+m+mi}{1}\PY{p}{)}
\end{Verbatim}
\end{tcolorbox}

    \hypertarget{splitting-data-and-test-and-train-sets-75-train-and-25-test}{%
\subsection{Splitting data and test and train sets (75-train and 25
test)}\label{splitting-data-and-test-and-train-sets-75-train-and-25-test}}

    \begin{tcolorbox}[breakable, size=fbox, boxrule=1pt, pad at break*=1mm,colback=cellbackground, colframe=cellborder]
\prompt{In}{incolor}{28}{\boxspacing}
\begin{Verbatim}[commandchars=\\\{\}]
\PY{n}{train\PYZus{}x}\PY{p}{,}\PY{n}{test\PYZus{}x}\PY{p}{,}\PY{n}{train\PYZus{}y}\PY{p}{,}\PY{n}{test\PYZus{}y} \PY{o}{=} \PY{n}{train\PYZus{}test\PYZus{}split}\PY{p}{(}\PY{n}{data\PYZus{}sec}\PY{p}{,}\PY{n}{data\PYZus{}sec\PYZus{}y}\PY{p}{,}\PY{n}{random\PYZus{}state} \PY{o}{=} \PY{l+m+mi}{0}\PY{p}{)}
\end{Verbatim}
\end{tcolorbox}

    \hypertarget{setting-up-naive-bayes-classifier-to-classify-whether-tumor-is-benign-or-malignant}{%
\subsection{Setting up Naive Bayes classifier to classify whether tumor
is benign or
malignant}\label{setting-up-naive-bayes-classifier-to-classify-whether-tumor-is-benign-or-malignant}}

\hypertarget{we-have-to-assume-that-the-distribution-of-the-data-is-gaussian-for-the-classifier-to-work-hence-the-naive}{%
\subsubsection{We have to assume that the distribution of the data is
Gaussian for the classifier to work (Hence the
`Naive')}\label{we-have-to-assume-that-the-distribution-of-the-data-is-gaussian-for-the-classifier-to-work-hence-the-naive}}

\hypertarget{to-this-end-we-have-used-guassiannb-function}{%
\subsubsection{To this end we have used GuassianNB( )
function}\label{to-this-end-we-have-used-guassiannb-function}}

    \begin{tcolorbox}[breakable, size=fbox, boxrule=1pt, pad at break*=1mm,colback=cellbackground, colframe=cellborder]
\prompt{In}{incolor}{29}{\boxspacing}
\begin{Verbatim}[commandchars=\\\{\}]
\PY{n}{headings} \PY{o}{=} \PY{n}{data\PYZus{}sec}\PY{o}{.}\PY{n}{columns}\PY{o}{.}\PY{n}{tolist}\PY{p}{(}\PY{p}{)}
\PY{n}{model} \PY{o}{=} \PY{n}{GaussianNB}\PY{p}{(}\PY{p}{)}
\PY{n}{model}\PY{o}{.}\PY{n}{fit}\PY{p}{(}\PY{n}{train\PYZus{}x}\PY{p}{[}\PY{n}{headings}\PY{p}{[}\PY{l+m+mi}{1}\PY{p}{:}\PY{l+m+mi}{3}\PY{p}{]}\PY{p}{]}\PY{p}{,}\PY{n}{train\PYZus{}y}\PY{p}{)}
\end{Verbatim}
\end{tcolorbox}

            \begin{tcolorbox}[breakable, size=fbox, boxrule=.5pt, pad at break*=1mm, opacityfill=0]
\prompt{Out}{outcolor}{29}{\boxspacing}
\begin{Verbatim}[commandchars=\\\{\}]
GaussianNB(priors=None, var\_smoothing=1e-09)
\end{Verbatim}
\end{tcolorbox}
        
    \hypertarget{error-metric-1-accuracy}{%
\subsection{Error Metric 1: Accuracy}\label{error-metric-1-accuracy}}

    \begin{tcolorbox}[breakable, size=fbox, boxrule=1pt, pad at break*=1mm,colback=cellbackground, colframe=cellborder]
\prompt{In}{incolor}{31}{\boxspacing}
\begin{Verbatim}[commandchars=\\\{\}]
\PY{n+nb}{print}\PY{p}{(}\PY{l+s+s2}{\PYZdq{}}\PY{l+s+s2}{The accuracy of the model on test data is found to be :}\PY{l+s+s2}{\PYZdq{}}\PY{p}{,}\PY{n}{model}\PY{o}{.}\PY{n}{score}\PY{p}{(}\PY{n}{test\PYZus{}x}\PY{p}{[}\PY{n}{headings}\PY{p}{[}\PY{l+m+mi}{1}\PY{p}{:}\PY{l+m+mi}{3}\PY{p}{]}\PY{p}{]}\PY{p}{,}\PY{n}{test\PYZus{}y}\PY{p}{)}\PY{p}{)}
\PY{n+nb}{print}\PY{p}{(}\PY{l+s+s2}{\PYZdq{}}\PY{l+s+s2}{The accuracy of the model on train data is found to be :}\PY{l+s+s2}{\PYZdq{}}\PY{p}{,}\PY{n}{model}\PY{o}{.}\PY{n}{score}\PY{p}{(}\PY{n}{train\PYZus{}x}\PY{p}{[}\PY{n}{headings}\PY{p}{[}\PY{l+m+mi}{1}\PY{p}{:}\PY{l+m+mi}{3}\PY{p}{]}\PY{p}{]}\PY{p}{,}\PY{n}{train\PYZus{}y}\PY{p}{)}\PY{p}{)}
\end{Verbatim}
\end{tcolorbox}

    \begin{Verbatim}[commandchars=\\\{\}]
The accuracy of the model on test data is found to be : 0.8951048951048951
The accuracy of the model on train data is found to be : 0.8849765258215962
    \end{Verbatim}

    \hypertarget{error-metric-2-precision}{%
\subsection{Error Metric 2: Precision}\label{error-metric-2-precision}}

\hypertarget{introduced-the-confusion-matrix-that-is-essential-to-calculate-the-necessary-metrics}{%
\subsubsection{Introduced the confusion matrix that is essential to
calculate the necessary
metrics}\label{introduced-the-confusion-matrix-that-is-essential-to-calculate-the-necessary-metrics}}

    \begin{tcolorbox}[breakable, size=fbox, boxrule=1pt, pad at break*=1mm,colback=cellbackground, colframe=cellborder]
\prompt{In}{incolor}{33}{\boxspacing}
\begin{Verbatim}[commandchars=\\\{\}]
\PY{n}{conf\PYZus{}mat} \PY{o}{=} \PY{n}{metrics}\PY{o}{.}\PY{n}{confusion\PYZus{}matrix}\PY{p}{(}\PY{n}{test\PYZus{}y}\PY{p}{,}\PY{n}{model}\PY{o}{.}\PY{n}{predict}\PY{p}{(}\PY{n}{test\PYZus{}x}\PY{p}{[}\PY{n}{headings}\PY{p}{[}\PY{l+m+mi}{1}\PY{p}{:}\PY{l+m+mi}{3}\PY{p}{]}\PY{p}{]}\PY{p}{)}\PY{p}{)}
\PY{n}{precision} \PY{o}{=} \PY{n}{conf\PYZus{}mat}\PY{p}{[}\PY{l+m+mi}{0}\PY{p}{]}\PY{p}{[}\PY{l+m+mi}{0}\PY{p}{]}\PY{o}{/}\PY{p}{(}\PY{n}{conf\PYZus{}mat}\PY{p}{[}\PY{l+m+mi}{0}\PY{p}{]}\PY{p}{[}\PY{l+m+mi}{0}\PY{p}{]}\PY{o}{+}\PY{n}{conf\PYZus{}mat}\PY{p}{[}\PY{l+m+mi}{0}\PY{p}{]}\PY{p}{[}\PY{l+m+mi}{1}\PY{p}{]}\PY{p}{)}
\PY{n+nb}{print}\PY{p}{(}\PY{l+s+s2}{\PYZdq{}}\PY{l+s+s2}{The precision of the model is found to be:}\PY{l+s+s2}{\PYZdq{}}\PY{p}{,}\PY{n}{precision}\PY{p}{)}
\end{Verbatim}
\end{tcolorbox}

    \begin{Verbatim}[commandchars=\\\{\}]
The precision of the model is found to be: 0.9333333333333333
    \end{Verbatim}

    \hypertarget{error-metric-3-recall}{%
\subsection{Error Metric 3: Recall}\label{error-metric-3-recall}}

    \begin{tcolorbox}[breakable, size=fbox, boxrule=1pt, pad at break*=1mm,colback=cellbackground, colframe=cellborder]
\prompt{In}{incolor}{34}{\boxspacing}
\begin{Verbatim}[commandchars=\\\{\}]
\PY{n}{recall} \PY{o}{=} \PY{n}{conf\PYZus{}mat}\PY{p}{[}\PY{l+m+mi}{0}\PY{p}{]}\PY{p}{[}\PY{l+m+mi}{0}\PY{p}{]}\PY{o}{/}\PY{p}{(}\PY{n}{conf\PYZus{}mat}\PY{p}{[}\PY{l+m+mi}{0}\PY{p}{]}\PY{p}{[}\PY{l+m+mi}{0}\PY{p}{]}\PY{o}{+}\PY{n}{conf\PYZus{}mat}\PY{p}{[}\PY{l+m+mi}{1}\PY{p}{]}\PY{p}{[}\PY{l+m+mi}{0}\PY{p}{]}\PY{p}{)}
\PY{n+nb}{print}\PY{p}{(}\PY{l+s+s2}{\PYZdq{}}\PY{l+s+s2}{The recall of the model is found to be:}\PY{l+s+s2}{\PYZdq{}}\PY{p}{,}\PY{n}{recall}\PY{p}{)}
\end{Verbatim}
\end{tcolorbox}

    \begin{Verbatim}[commandchars=\\\{\}]
The recall of the model is found to be: 0.9032258064516129
    \end{Verbatim}

    \hypertarget{error-metric-4-f1-score-the-most-accurate-metric-to-decide-on-how-well-the-model-is-doing}{%
\subsection{Error Metric 4: F1 score (The most accurate metric to decide
on how well the model is
doing}\label{error-metric-4-f1-score-the-most-accurate-metric-to-decide-on-how-well-the-model-is-doing}}

    \begin{tcolorbox}[breakable, size=fbox, boxrule=1pt, pad at break*=1mm,colback=cellbackground, colframe=cellborder]
\prompt{In}{incolor}{35}{\boxspacing}
\begin{Verbatim}[commandchars=\\\{\}]
\PY{n}{f1\PYZus{}score} \PY{o}{=} \PY{l+m+mi}{2}\PY{o}{*}\PY{p}{(}\PY{p}{(}\PY{n}{precision}\PY{o}{*}\PY{n}{recall}\PY{p}{)}\PY{o}{/}\PY{p}{(}\PY{n}{precision}\PY{o}{+}\PY{n}{recall}\PY{p}{)}\PY{p}{)}
\PY{n+nb}{print}\PY{p}{(}\PY{l+s+s2}{\PYZdq{}}\PY{l+s+s2}{The F1 score of the model is :}\PY{l+s+s2}{\PYZdq{}}\PY{p}{,}\PY{n}{f1\PYZus{}score}\PY{p}{)}
\end{Verbatim}
\end{tcolorbox}

    \begin{Verbatim}[commandchars=\\\{\}]
The F1 score of the model is : 0.9180327868852459
    \end{Verbatim}

    \hypertarget{plotting-error-metrics}{%
\section{Plotting error metrics}\label{plotting-error-metrics}}

\hypertarget{plot-1-precision-vs-recall}{%
\subsection{Plot 1: Precision vs
Recall}\label{plot-1-precision-vs-recall}}

    \begin{tcolorbox}[breakable, size=fbox, boxrule=1pt, pad at break*=1mm,colback=cellbackground, colframe=cellborder]
\prompt{In}{incolor}{36}{\boxspacing}
\begin{Verbatim}[commandchars=\\\{\}]
\PY{n}{disp} \PY{o}{=} \PY{n}{plot\PYZus{}precision\PYZus{}recall\PYZus{}curve}\PY{p}{(}\PY{n}{model}\PY{p}{,} \PY{n}{test\PYZus{}x}\PY{p}{[}\PY{n}{headings}\PY{p}{[}\PY{l+m+mi}{1}\PY{p}{:}\PY{l+m+mi}{3}\PY{p}{]}\PY{p}{]}\PY{p}{,}\PY{n}{test\PYZus{}y}\PY{p}{)}
\end{Verbatim}
\end{tcolorbox}

    \begin{center}
    \adjustimage{max size={0.9\linewidth}{0.9\paperheight}}{output_21_0.png}
    \end{center}
    { \hspace*{\fill} \\}
    
    \hypertarget{plot-2-the-confusion-matrix}{%
\subsection{Plot 2: The confusion
matrix}\label{plot-2-the-confusion-matrix}}

    \begin{tcolorbox}[breakable, size=fbox, boxrule=1pt, pad at break*=1mm,colback=cellbackground, colframe=cellborder]
\prompt{In}{incolor}{37}{\boxspacing}
\begin{Verbatim}[commandchars=\\\{\}]
\PY{n}{plot\PYZus{}confusion\PYZus{}matrix}\PY{p}{(}\PY{n}{model}\PY{p}{,}\PY{n}{test\PYZus{}x}\PY{p}{[}\PY{n}{headings}\PY{p}{[}\PY{l+m+mi}{1}\PY{p}{:}\PY{l+m+mi}{3}\PY{p}{]}\PY{p}{]}\PY{p}{,}\PY{n}{test\PYZus{}y}\PY{p}{)}
\end{Verbatim}
\end{tcolorbox}

            \begin{tcolorbox}[breakable, size=fbox, boxrule=.5pt, pad at break*=1mm, opacityfill=0]
\prompt{Out}{outcolor}{37}{\boxspacing}
\begin{Verbatim}[commandchars=\\\{\}]
<sklearn.metrics.\_plot.confusion\_matrix.ConfusionMatrixDisplay at 0x2476d1183c8>
\end{Verbatim}
\end{tcolorbox}
        
    \begin{center}
    \adjustimage{max size={0.9\linewidth}{0.9\paperheight}}{output_23_1.png}
    \end{center}
    { \hspace*{\fill} \\}
    
    \hypertarget{naive-bayes-classification-on-nominal-data}{%
\section{Naive Bayes classification on Nominal
data}\label{naive-bayes-classification-on-nominal-data}}

\hypertarget{loading-data-from-sklearn}{%
\subsection{Loading data from sklearn}\label{loading-data-from-sklearn}}

    \begin{tcolorbox}[breakable, size=fbox, boxrule=1pt, pad at break*=1mm,colback=cellbackground, colframe=cellborder]
\prompt{In}{incolor}{38}{\boxspacing}
\begin{Verbatim}[commandchars=\\\{\}]
\PY{n}{data} \PY{o}{=} \PY{n}{fetch\PYZus{}20newsgroups}\PY{p}{(}\PY{p}{)}
\end{Verbatim}
\end{tcolorbox}

    \hypertarget{selecting-specific-categories-as-features-and-splitting-data-and-test-and-train-sets}{%
\subsection{Selecting specific categories as features and splitting data
and test and train
sets}\label{selecting-specific-categories-as-features-and-splitting-data-and-test-and-train-sets}}

    \begin{tcolorbox}[breakable, size=fbox, boxrule=1pt, pad at break*=1mm,colback=cellbackground, colframe=cellborder]
\prompt{In}{incolor}{39}{\boxspacing}
\begin{Verbatim}[commandchars=\\\{\}]
\PY{n}{categories} \PY{o}{=} \PY{p}{[}\PY{l+s+s1}{\PYZsq{}}\PY{l+s+s1}{talk.religion.misc}\PY{l+s+s1}{\PYZsq{}}\PY{p}{,} \PY{l+s+s1}{\PYZsq{}}\PY{l+s+s1}{soc.religion.christian}\PY{l+s+s1}{\PYZsq{}}\PY{p}{,}
              \PY{l+s+s1}{\PYZsq{}}\PY{l+s+s1}{sci.space}\PY{l+s+s1}{\PYZsq{}}\PY{p}{,} \PY{l+s+s1}{\PYZsq{}}\PY{l+s+s1}{comp.graphics}\PY{l+s+s1}{\PYZsq{}}\PY{p}{]}
\PY{n}{train} \PY{o}{=} \PY{n}{fetch\PYZus{}20newsgroups}\PY{p}{(}\PY{n}{subset}\PY{o}{=}\PY{l+s+s1}{\PYZsq{}}\PY{l+s+s1}{train}\PY{l+s+s1}{\PYZsq{}}\PY{p}{,} \PY{n}{categories}\PY{o}{=}\PY{n}{categories}\PY{p}{)}
\PY{n}{test} \PY{o}{=} \PY{n}{fetch\PYZus{}20newsgroups}\PY{p}{(}\PY{n}{subset}\PY{o}{=}\PY{l+s+s1}{\PYZsq{}}\PY{l+s+s1}{test}\PY{l+s+s1}{\PYZsq{}}\PY{p}{,} \PY{n}{categories}\PY{o}{=}\PY{n}{categories}\PY{p}{)}
\end{Verbatim}
\end{tcolorbox}

    \hypertarget{preview-of-the-data}{%
\subsection{Preview of the data}\label{preview-of-the-data}}

    \begin{tcolorbox}[breakable, size=fbox, boxrule=1pt, pad at break*=1mm,colback=cellbackground, colframe=cellborder]
\prompt{In}{incolor}{40}{\boxspacing}
\begin{Verbatim}[commandchars=\\\{\}]
\PY{n+nb}{print}\PY{p}{(}\PY{n}{train}\PY{o}{.}\PY{n}{data}\PY{p}{[}\PY{l+m+mi}{5}\PY{p}{]}\PY{p}{)}
\end{Verbatim}
\end{tcolorbox}

    \begin{Verbatim}[commandchars=\\\{\}]
From: dmcgee@uluhe.soest.hawaii.edu (Don McGee)
Subject: Federal Hearing
Originator: dmcgee@uluhe
Organization: School of Ocean and Earth Science and Technology
Distribution: usa
Lines: 10


Fact or rumor{\ldots}?  Madalyn Murray O'Hare an atheist who eliminated the
use of the bible reading and prayer in public schools 15 years ago is now
going to appear before the FCC with a petition to stop the reading of the
Gospel on the airways of America.  And she is also campaigning to remove
Christmas programs, songs, etc from the public schools.  If it is true
then mail to Federal Communications Commission 1919 H Street Washington DC
20054 expressing your opposition to her request.  Reference Petition number

2493.

    \end{Verbatim}

    \hypertarget{setting-up-naive-bayes-classification-to-classify-into-what-category-of-news-it-is}{%
\subsection{Setting up Naive Bayes classification to classify into what
category of news it
is}\label{setting-up-naive-bayes-classification-to-classify-into-what-category-of-news-it-is}}

\hypertarget{we-first-vectorize-the-articles-using-tf-idf-that-creates-a-feature-vector-oout-of-the-frequency-of-words-in-an-article}{%
\subsubsection{We first vectorize the articles using tf-idf that creates
a feature vector oout of the frequency of words in an
article}\label{we-first-vectorize-the-articles-using-tf-idf-that-creates-a-feature-vector-oout-of-the-frequency-of-words-in-an-article}}

    \begin{tcolorbox}[breakable, size=fbox, boxrule=1pt, pad at break*=1mm,colback=cellbackground, colframe=cellborder]
\prompt{In}{incolor}{41}{\boxspacing}
\begin{Verbatim}[commandchars=\\\{\}]
\PY{n}{model} \PY{o}{=} \PY{n}{make\PYZus{}pipeline}\PY{p}{(}\PY{n}{TfidfVectorizer}\PY{p}{(}\PY{p}{)}\PY{p}{,} \PY{n}{MultinomialNB}\PY{p}{(}\PY{p}{)}\PY{p}{)}
\PY{n}{model}\PY{o}{.}\PY{n}{fit}\PY{p}{(}\PY{n}{train}\PY{o}{.}\PY{n}{data}\PY{p}{,} \PY{n}{train}\PY{o}{.}\PY{n}{target}\PY{p}{)}
\end{Verbatim}
\end{tcolorbox}

            \begin{tcolorbox}[breakable, size=fbox, boxrule=.5pt, pad at break*=1mm, opacityfill=0]
\prompt{Out}{outcolor}{41}{\boxspacing}
\begin{Verbatim}[commandchars=\\\{\}]
Pipeline(memory=None,
         steps=[('tfidfvectorizer',
                 TfidfVectorizer(analyzer='word', binary=False,
                                 decode\_error='strict',
                                 dtype=<class 'numpy.float64'>,
                                 encoding='utf-8', input='content',
                                 lowercase=True, max\_df=1.0, max\_features=None,
                                 min\_df=1, ngram\_range=(1, 1), norm='l2',
                                 preprocessor=None, smooth\_idf=True,
                                 stop\_words=None, strip\_accents=None,
                                 sublinear\_tf=False,
                                 token\_pattern='(?u)\textbackslash{}\textbackslash{}b\textbackslash{}\textbackslash{}w\textbackslash{}\textbackslash{}w+\textbackslash{}\textbackslash{}b',
                                 tokenizer=None, use\_idf=True,
                                 vocabulary=None)),
                ('multinomialnb',
                 MultinomialNB(alpha=1.0, class\_prior=None, fit\_prior=True))],
         verbose=False)
\end{Verbatim}
\end{tcolorbox}
        
    \hypertarget{predicting-class-labels}{%
\subsection{Predicting class labels}\label{predicting-class-labels}}

    \begin{tcolorbox}[breakable, size=fbox, boxrule=1pt, pad at break*=1mm,colback=cellbackground, colframe=cellborder]
\prompt{In}{incolor}{44}{\boxspacing}
\begin{Verbatim}[commandchars=\\\{\}]
\PY{n}{label} \PY{o}{=} \PY{n}{model}\PY{o}{.}\PY{n}{predict}\PY{p}{(}\PY{n}{test}\PY{o}{.}\PY{n}{data}\PY{p}{)}
\end{Verbatim}
\end{tcolorbox}

    \hypertarget{error-metric-1-accuracy}{%
\subsection{Error metric 1: Accuracy}\label{error-metric-1-accuracy}}

    \begin{tcolorbox}[breakable, size=fbox, boxrule=1pt, pad at break*=1mm,colback=cellbackground, colframe=cellborder]
\prompt{In}{incolor}{45}{\boxspacing}
\begin{Verbatim}[commandchars=\\\{\}]
\PY{n+nb}{print}\PY{p}{(}\PY{l+s+s2}{\PYZdq{}}\PY{l+s+s2}{The accuracy of the model on test data is found to be :}\PY{l+s+s2}{\PYZdq{}}\PY{p}{,}\PY{n}{model}\PY{o}{.}\PY{n}{score}\PY{p}{(}\PY{n}{test}\PY{o}{.}\PY{n}{data}\PY{p}{,}\PY{n}{test}\PY{o}{.}\PY{n}{target}\PY{p}{)}\PY{p}{)}
\PY{n+nb}{print}\PY{p}{(}\PY{l+s+s2}{\PYZdq{}}\PY{l+s+s2}{The accuracy of the model on train data is found to be :}\PY{l+s+s2}{\PYZdq{}}\PY{p}{,}\PY{n}{model}\PY{o}{.}\PY{n}{score}\PY{p}{(}\PY{n}{train}\PY{o}{.}\PY{n}{data}\PY{p}{,}\PY{n}{train}\PY{o}{.}\PY{n}{target}\PY{p}{)}\PY{p}{)}
\end{Verbatim}
\end{tcolorbox}

    \begin{Verbatim}[commandchars=\\\{\}]
The accuracy of the model on test data is found to be : 0.8016759776536313
The accuracy of the model on train data is found to be : 0.8959591267998143
    \end{Verbatim}

    \hypertarget{error-metric-2-precision}{%
\subsection{Error Metric 2: Precision}\label{error-metric-2-precision}}

\hypertarget{introduced-the-confusion-matrix-that-is-essential-to-calculate-the-necessary-metrics}{%
\subsubsection{Introduced the confusion matrix that is essential to
calculate the necessary
metrics}\label{introduced-the-confusion-matrix-that-is-essential-to-calculate-the-necessary-metrics}}

    \begin{tcolorbox}[breakable, size=fbox, boxrule=1pt, pad at break*=1mm,colback=cellbackground, colframe=cellborder]
\prompt{In}{incolor}{53}{\boxspacing}
\begin{Verbatim}[commandchars=\\\{\}]
\PY{n}{conf\PYZus{}mat} \PY{o}{=} \PY{n}{metrics}\PY{o}{.}\PY{n}{confusion\PYZus{}matrix}\PY{p}{(}\PY{n}{test}\PY{o}{.}\PY{n}{target}\PY{p}{,}\PY{n}{model}\PY{o}{.}\PY{n}{predict}\PY{p}{(}\PY{n}{test}\PY{o}{.}\PY{n}{data}\PY{p}{)}\PY{p}{)}
\PY{n}{precision\PYZus{}1} \PY{o}{=} \PY{n}{conf\PYZus{}mat}\PY{p}{[}\PY{l+m+mi}{0}\PY{p}{]}\PY{p}{[}\PY{l+m+mi}{0}\PY{p}{]}\PY{o}{/}\PY{p}{(}\PY{n}{conf\PYZus{}mat}\PY{p}{[}\PY{l+m+mi}{0}\PY{p}{]}\PY{p}{[}\PY{l+m+mi}{0}\PY{p}{]}\PY{o}{+}\PY{n}{conf\PYZus{}mat}\PY{p}{[}\PY{l+m+mi}{0}\PY{p}{]}\PY{p}{[}\PY{l+m+mi}{0}\PY{p}{]}\PY{o}{+}\PY{n}{conf\PYZus{}mat}\PY{p}{[}\PY{l+m+mi}{0}\PY{p}{]}\PY{p}{[}\PY{l+m+mi}{1}\PY{p}{]}\PY{o}{+}\PY{n}{conf\PYZus{}mat}\PY{p}{[}\PY{l+m+mi}{0}\PY{p}{]}\PY{p}{[}\PY{l+m+mi}{2}\PY{p}{]}\PY{p}{)}
\PY{n}{precision\PYZus{}2} \PY{o}{=} \PY{n}{conf\PYZus{}mat}\PY{p}{[}\PY{l+m+mi}{1}\PY{p}{]}\PY{p}{[}\PY{l+m+mi}{1}\PY{p}{]}\PY{o}{/}\PY{p}{(}\PY{n}{conf\PYZus{}mat}\PY{p}{[}\PY{l+m+mi}{1}\PY{p}{]}\PY{p}{[}\PY{l+m+mi}{1}\PY{p}{]}\PY{o}{+}\PY{n}{conf\PYZus{}mat}\PY{p}{[}\PY{l+m+mi}{1}\PY{p}{]}\PY{p}{[}\PY{l+m+mi}{0}\PY{p}{]}\PY{o}{+}\PY{n}{conf\PYZus{}mat}\PY{p}{[}\PY{l+m+mi}{1}\PY{p}{]}\PY{p}{[}\PY{l+m+mi}{2}\PY{p}{]}\PY{o}{+}\PY{n}{conf\PYZus{}mat}\PY{p}{[}\PY{l+m+mi}{1}\PY{p}{]}\PY{p}{[}\PY{l+m+mi}{3}\PY{p}{]}\PY{p}{)}
\PY{n}{precision\PYZus{}3} \PY{o}{=} \PY{n}{conf\PYZus{}mat}\PY{p}{[}\PY{l+m+mi}{2}\PY{p}{]}\PY{p}{[}\PY{l+m+mi}{2}\PY{p}{]}\PY{o}{/}\PY{p}{(}\PY{n}{conf\PYZus{}mat}\PY{p}{[}\PY{l+m+mi}{2}\PY{p}{]}\PY{p}{[}\PY{l+m+mi}{2}\PY{p}{]}\PY{o}{+}\PY{n}{conf\PYZus{}mat}\PY{p}{[}\PY{l+m+mi}{2}\PY{p}{]}\PY{p}{[}\PY{l+m+mi}{0}\PY{p}{]}\PY{o}{+}\PY{n}{conf\PYZus{}mat}\PY{p}{[}\PY{l+m+mi}{2}\PY{p}{]}\PY{p}{[}\PY{l+m+mi}{1}\PY{p}{]}\PY{o}{+}\PY{n}{conf\PYZus{}mat}\PY{p}{[}\PY{l+m+mi}{2}\PY{p}{]}\PY{p}{[}\PY{l+m+mi}{3}\PY{p}{]}\PY{p}{)}
\PY{n}{precision\PYZus{}4} \PY{o}{=} \PY{n}{conf\PYZus{}mat}\PY{p}{[}\PY{l+m+mi}{3}\PY{p}{]}\PY{p}{[}\PY{l+m+mi}{3}\PY{p}{]}\PY{o}{/}\PY{p}{(}\PY{n}{conf\PYZus{}mat}\PY{p}{[}\PY{l+m+mi}{3}\PY{p}{]}\PY{p}{[}\PY{l+m+mi}{3}\PY{p}{]}\PY{o}{+}\PY{n}{conf\PYZus{}mat}\PY{p}{[}\PY{l+m+mi}{3}\PY{p}{]}\PY{p}{[}\PY{l+m+mi}{0}\PY{p}{]}\PY{o}{+}\PY{n}{conf\PYZus{}mat}\PY{p}{[}\PY{l+m+mi}{3}\PY{p}{]}\PY{p}{[}\PY{l+m+mi}{1}\PY{p}{]}\PY{o}{+}\PY{n}{conf\PYZus{}mat}\PY{p}{[}\PY{l+m+mi}{3}\PY{p}{]}\PY{p}{[}\PY{l+m+mi}{2}\PY{p}{]}\PY{p}{)}
\PY{n+nb}{print}\PY{p}{(}\PY{l+s+s2}{\PYZdq{}}\PY{l+s+s2}{The precision of the model for class 1 is found to be:}\PY{l+s+s2}{\PYZdq{}}\PY{p}{,}\PY{n}{precision\PYZus{}1}\PY{p}{)}
\PY{n+nb}{print}\PY{p}{(}\PY{l+s+s2}{\PYZdq{}}\PY{l+s+s2}{The precision of the model for class 2 is found to be:}\PY{l+s+s2}{\PYZdq{}}\PY{p}{,}\PY{n}{precision\PYZus{}2}\PY{p}{)}
\PY{n+nb}{print}\PY{p}{(}\PY{l+s+s2}{\PYZdq{}}\PY{l+s+s2}{The precision of the model for class 3 is found to be:}\PY{l+s+s2}{\PYZdq{}}\PY{p}{,}\PY{n}{precision\PYZus{}3}\PY{p}{)}
\PY{n+nb}{print}\PY{p}{(}\PY{l+s+s2}{\PYZdq{}}\PY{l+s+s2}{The precision of the model for class 4 is found to be:}\PY{l+s+s2}{\PYZdq{}}\PY{p}{,}\PY{n}{precision\PYZus{}4}\PY{p}{)}
\end{Verbatim}
\end{tcolorbox}

    \begin{Verbatim}[commandchars=\\\{\}]
The precision of the model for class 1 is found to be: 0.4693042291950887
The precision of the model for class 2 is found to be: 0.9238578680203046
The precision of the model for class 3 is found to be: 0.9849246231155779
The precision of the model for class 4 is found to be: 0.19123505976095617
    \end{Verbatim}

    \hypertarget{error-metric-3-recall}{%
\subsection{Error Metric 3: Recall}\label{error-metric-3-recall}}

    \begin{tcolorbox}[breakable, size=fbox, boxrule=1pt, pad at break*=1mm,colback=cellbackground, colframe=cellborder]
\prompt{In}{incolor}{54}{\boxspacing}
\begin{Verbatim}[commandchars=\\\{\}]
\PY{n}{rec\PYZus{}1} \PY{o}{=} \PY{n}{conf\PYZus{}mat}\PY{p}{[}\PY{l+m+mi}{0}\PY{p}{]}\PY{p}{[}\PY{l+m+mi}{0}\PY{p}{]}\PY{o}{/}\PY{p}{(}\PY{n}{conf\PYZus{}mat}\PY{p}{[}\PY{l+m+mi}{0}\PY{p}{]}\PY{p}{[}\PY{l+m+mi}{0}\PY{p}{]}\PY{o}{+}\PY{n}{conf\PYZus{}mat}\PY{p}{[}\PY{l+m+mi}{1}\PY{p}{]}\PY{p}{[}\PY{l+m+mi}{0}\PY{p}{]}\PY{o}{+}\PY{n}{conf\PYZus{}mat}\PY{p}{[}\PY{l+m+mi}{2}\PY{p}{]}\PY{p}{[}\PY{l+m+mi}{0}\PY{p}{]}\PY{o}{+}\PY{n}{conf\PYZus{}mat}\PY{p}{[}\PY{l+m+mi}{3}\PY{p}{]}\PY{p}{[}\PY{l+m+mi}{0}\PY{p}{]}\PY{p}{)}
\PY{n}{rec\PYZus{}2} \PY{o}{=} \PY{n}{conf\PYZus{}mat}\PY{p}{[}\PY{l+m+mi}{1}\PY{p}{]}\PY{p}{[}\PY{l+m+mi}{1}\PY{p}{]}\PY{o}{/}\PY{p}{(}\PY{n}{conf\PYZus{}mat}\PY{p}{[}\PY{l+m+mi}{1}\PY{p}{]}\PY{p}{[}\PY{l+m+mi}{1}\PY{p}{]}\PY{o}{+}\PY{n}{conf\PYZus{}mat}\PY{p}{[}\PY{l+m+mi}{2}\PY{p}{]}\PY{p}{[}\PY{l+m+mi}{1}\PY{p}{]}\PY{o}{+}\PY{n}{conf\PYZus{}mat}\PY{p}{[}\PY{l+m+mi}{3}\PY{p}{]}\PY{p}{[}\PY{l+m+mi}{1}\PY{p}{]}\PY{o}{+}\PY{n}{conf\PYZus{}mat}\PY{p}{[}\PY{l+m+mi}{0}\PY{p}{]}\PY{p}{[}\PY{l+m+mi}{1}\PY{p}{]}\PY{p}{)}
\PY{n}{rec\PYZus{}3} \PY{o}{=} \PY{n}{conf\PYZus{}mat}\PY{p}{[}\PY{l+m+mi}{2}\PY{p}{]}\PY{p}{[}\PY{l+m+mi}{2}\PY{p}{]}\PY{o}{/}\PY{p}{(}\PY{n}{conf\PYZus{}mat}\PY{p}{[}\PY{l+m+mi}{2}\PY{p}{]}\PY{p}{[}\PY{l+m+mi}{2}\PY{p}{]}\PY{o}{+}\PY{n}{conf\PYZus{}mat}\PY{p}{[}\PY{l+m+mi}{0}\PY{p}{]}\PY{p}{[}\PY{l+m+mi}{2}\PY{p}{]}\PY{o}{+}\PY{n}{conf\PYZus{}mat}\PY{p}{[}\PY{l+m+mi}{1}\PY{p}{]}\PY{p}{[}\PY{l+m+mi}{2}\PY{p}{]}\PY{o}{+}\PY{n}{conf\PYZus{}mat}\PY{p}{[}\PY{l+m+mi}{3}\PY{p}{]}\PY{p}{[}\PY{l+m+mi}{2}\PY{p}{]}\PY{p}{)}
\PY{n}{rec\PYZus{}4} \PY{o}{=} \PY{n}{conf\PYZus{}mat}\PY{p}{[}\PY{l+m+mi}{3}\PY{p}{]}\PY{p}{[}\PY{l+m+mi}{3}\PY{p}{]}\PY{o}{/}\PY{p}{(}\PY{n}{conf\PYZus{}mat}\PY{p}{[}\PY{l+m+mi}{3}\PY{p}{]}\PY{p}{[}\PY{l+m+mi}{3}\PY{p}{]}\PY{o}{+}\PY{n}{conf\PYZus{}mat}\PY{p}{[}\PY{l+m+mi}{0}\PY{p}{]}\PY{p}{[}\PY{l+m+mi}{3}\PY{p}{]}\PY{o}{+}\PY{n}{conf\PYZus{}mat}\PY{p}{[}\PY{l+m+mi}{1}\PY{p}{]}\PY{p}{[}\PY{l+m+mi}{3}\PY{p}{]}\PY{o}{+}\PY{n}{conf\PYZus{}mat}\PY{p}{[}\PY{l+m+mi}{2}\PY{p}{]}\PY{p}{[}\PY{l+m+mi}{3}\PY{p}{]}\PY{p}{)}
\PY{n+nb}{print}\PY{p}{(}\PY{l+s+s2}{\PYZdq{}}\PY{l+s+s2}{The recall of the model for class 1 is found to be:}\PY{l+s+s2}{\PYZdq{}}\PY{p}{,}\PY{n}{rec\PYZus{}1}\PY{p}{)}
\PY{n+nb}{print}\PY{p}{(}\PY{l+s+s2}{\PYZdq{}}\PY{l+s+s2}{The recall of the model for class 2 is found to be:}\PY{l+s+s2}{\PYZdq{}}\PY{p}{,}\PY{n}{rec\PYZus{}2}\PY{p}{)}
\PY{n+nb}{print}\PY{p}{(}\PY{l+s+s2}{\PYZdq{}}\PY{l+s+s2}{The recall of the model for class 3 is found to be:}\PY{l+s+s2}{\PYZdq{}}\PY{p}{,}\PY{n}{rec\PYZus{}3}\PY{p}{)}
\PY{n+nb}{print}\PY{p}{(}\PY{l+s+s2}{\PYZdq{}}\PY{l+s+s2}{The recall of the model for class 4 is found to be:}\PY{l+s+s2}{\PYZdq{}}\PY{p}{,}\PY{n}{rec\PYZus{}4}\PY{p}{)}
\end{Verbatim}
\end{tcolorbox}

    \begin{Verbatim}[commandchars=\\\{\}]
The recall of the model for class 1 is found to be: 0.9690140845070423
The recall of the model for class 2 is found to be: 0.9238578680203046
The recall of the model for class 3 is found to be: 0.6173228346456693
The recall of the model for class 4 is found to be: 1.0
    \end{Verbatim}

    \hypertarget{error-metric-4-f1-score}{%
\subsection{Error Metric 4: F1 Score}\label{error-metric-4-f1-score}}

    \begin{tcolorbox}[breakable, size=fbox, boxrule=1pt, pad at break*=1mm,colback=cellbackground, colframe=cellborder]
\prompt{In}{incolor}{56}{\boxspacing}
\begin{Verbatim}[commandchars=\\\{\}]
\PY{n}{f1\PYZus{}score\PYZus{}1} \PY{o}{=} \PY{l+m+mi}{2}\PY{o}{*}\PY{p}{(}\PY{p}{(}\PY{n}{precision\PYZus{}1}\PY{o}{*}\PY{n}{rec\PYZus{}1}\PY{p}{)}\PY{o}{/}\PY{p}{(}\PY{n}{precision\PYZus{}1}\PY{o}{+}\PY{n}{rec\PYZus{}1}\PY{p}{)}\PY{p}{)}
\PY{n}{f1\PYZus{}score\PYZus{}2} \PY{o}{=} \PY{l+m+mi}{2}\PY{o}{*}\PY{p}{(}\PY{p}{(}\PY{n}{precision\PYZus{}2}\PY{o}{*}\PY{n}{rec\PYZus{}2}\PY{p}{)}\PY{o}{/}\PY{p}{(}\PY{n}{precision\PYZus{}2}\PY{o}{+}\PY{n}{rec\PYZus{}2}\PY{p}{)}\PY{p}{)}
\PY{n}{f1\PYZus{}score\PYZus{}3} \PY{o}{=} \PY{l+m+mi}{2}\PY{o}{*}\PY{p}{(}\PY{p}{(}\PY{n}{precision\PYZus{}3}\PY{o}{*}\PY{n}{rec\PYZus{}3}\PY{p}{)}\PY{o}{/}\PY{p}{(}\PY{n}{precision\PYZus{}3}\PY{o}{+}\PY{n}{rec\PYZus{}3}\PY{p}{)}\PY{p}{)}
\PY{n}{f1\PYZus{}score\PYZus{}4} \PY{o}{=} \PY{l+m+mi}{2}\PY{o}{*}\PY{p}{(}\PY{p}{(}\PY{n}{precision\PYZus{}4}\PY{o}{*}\PY{n}{rec\PYZus{}4}\PY{p}{)}\PY{o}{/}\PY{p}{(}\PY{n}{precision\PYZus{}4}\PY{o}{+}\PY{n}{rec\PYZus{}4}\PY{p}{)}\PY{p}{)}

\PY{n+nb}{print}\PY{p}{(}\PY{l+s+s2}{\PYZdq{}}\PY{l+s+s2}{The F1 score of the model for class 1 is :}\PY{l+s+s2}{\PYZdq{}}\PY{p}{,}\PY{n}{f1\PYZus{}score\PYZus{}1}\PY{p}{)}
\PY{n+nb}{print}\PY{p}{(}\PY{l+s+s2}{\PYZdq{}}\PY{l+s+s2}{The F1 score of the model for class 2 is :}\PY{l+s+s2}{\PYZdq{}}\PY{p}{,}\PY{n}{f1\PYZus{}score\PYZus{}2}\PY{p}{)}
\PY{n+nb}{print}\PY{p}{(}\PY{l+s+s2}{\PYZdq{}}\PY{l+s+s2}{The F1 score of the model for class 3 is :}\PY{l+s+s2}{\PYZdq{}}\PY{p}{,}\PY{n}{f1\PYZus{}score\PYZus{}3}\PY{p}{)}
\PY{n+nb}{print}\PY{p}{(}\PY{l+s+s2}{\PYZdq{}}\PY{l+s+s2}{The F1 score of the model for class 4 is :}\PY{l+s+s2}{\PYZdq{}}\PY{p}{,}\PY{n}{f1\PYZus{}score\PYZus{}4}\PY{p}{)}
\end{Verbatim}
\end{tcolorbox}

    \begin{Verbatim}[commandchars=\\\{\}]
The F1 score of the model for class 1 is : 0.6323529411764706
The F1 score of the model for class 2 is : 0.9238578680203046
The F1 score of the model for class 3 is : 0.7589545014520812
The F1 score of the model for class 4 is : 0.3210702341137123
    \end{Verbatim}

    \hypertarget{plotting-error-metrics}{%
\section{Plotting error metrics}\label{plotting-error-metrics}}

\hypertarget{plot-1-confusion-matrix}{%
\subsection{Plot 1: Confusion matrix}\label{plot-1-confusion-matrix}}

    \begin{tcolorbox}[breakable, size=fbox, boxrule=1pt, pad at break*=1mm,colback=cellbackground, colframe=cellborder]
\prompt{In}{incolor}{58}{\boxspacing}
\begin{Verbatim}[commandchars=\\\{\}]
\PY{n}{mat} \PY{o}{=} \PY{n}{confusion\PYZus{}matrix}\PY{p}{(}\PY{n}{test}\PY{o}{.}\PY{n}{target}\PY{p}{,} \PY{n}{labels}\PY{p}{)}
\PY{n}{sns}\PY{o}{.}\PY{n}{heatmap}\PY{p}{(}\PY{n}{mat}\PY{o}{.}\PY{n}{T}\PY{p}{,} \PY{n}{square}\PY{o}{=}\PY{k+kc}{True}\PY{p}{,} \PY{n}{annot}\PY{o}{=}\PY{k+kc}{True}\PY{p}{,} \PY{n}{fmt}\PY{o}{=}\PY{l+s+s1}{\PYZsq{}}\PY{l+s+s1}{d}\PY{l+s+s1}{\PYZsq{}}\PY{p}{,} \PY{n}{cbar}\PY{o}{=}\PY{k+kc}{False}\PY{p}{,}
            \PY{n}{xticklabels}\PY{o}{=}\PY{n}{train}\PY{o}{.}\PY{n}{target\PYZus{}names}\PY{p}{,} \PY{n}{yticklabels}\PY{o}{=}\PY{n}{train}\PY{o}{.}\PY{n}{target\PYZus{}names}\PY{p}{)}
\PY{n}{plt}\PY{o}{.}\PY{n}{xlabel}\PY{p}{(}\PY{l+s+s1}{\PYZsq{}}\PY{l+s+s1}{true label}\PY{l+s+s1}{\PYZsq{}}\PY{p}{)}
\PY{n}{plt}\PY{o}{.}\PY{n}{ylabel}\PY{p}{(}\PY{l+s+s1}{\PYZsq{}}\PY{l+s+s1}{predicted label}\PY{l+s+s1}{\PYZsq{}}\PY{p}{)}\PY{p}{;}
\end{Verbatim}
\end{tcolorbox}

    \begin{center}
    \adjustimage{max size={0.9\linewidth}{0.9\paperheight}}{output_43_0.png}
    \end{center}
    { \hspace*{\fill} \\}
    
    End of notebook

    \begin{tcolorbox}[breakable, size=fbox, boxrule=1pt, pad at break*=1mm,colback=cellbackground, colframe=cellborder]
\prompt{In}{incolor}{ }{\boxspacing}
\begin{Verbatim}[commandchars=\\\{\}]

\end{Verbatim}
\end{tcolorbox}


    % Add a bibliography block to the postdoc
    
    
    
\end{document}
